\section{Datasets}
The datasets that will be used in this project are nine and they are divided into three categories: road networks, social networks, and biological networks.
For each of these categories, three datasets of increasing size were selected: one small, one medium, and one large. \\
The datasets, with their main properties (i.e., the number of nodes, the number of edges, and the type of graph), are presented in Table \ref{tab:datasets}. 
\begin{table}[h]
    \centering
    \begin{tabular}{|c|c|c|c|c|}
        \hline
        \textbf{\scalebox{0.87}{\large{Network}}} & \scalebox{0.87}{\large$\bm{|V|}$} & \scalebox{0.87}{\large$\bm{|E|}$} & \textbf{\scalebox{0.87}{\large{Type}}} \\
        \hline
        Pennsylvania \cite{nr} & 1,088,092 & 1,541,898 & Undirected \\
        \hline
        Padua (province) \cite{padua2025} & 122,680 & 304,184 & Directed \\
        \hline
        Hong Kong (city) \cite{hongkong2025} & 43,620 & 91,542 & Directed \\
        \hline
        Italian Covid-19 Retweet Network \cite{mesina2024italian} & 221,574 & 800,000 & Directed \\
        \hline
        Twitch Gamers \cite{rozemberczki2021twitch} & 168,114 & 6,797,557 & Undirected \\
        \hline
        GitHub Developers \cite{rozemberczki2019multiscale} & 37,700 & 289,003 & Undirected \\
        \hline
        Mus Musculus Protein Interactions \cite{szklarczyk2023string} & 20,969 & 800,000 & Undirected \\
        \hline
        Saccharomyces cerevisiae Protein Interactions \cite{szklarczyk2023string} & 5,786 & 100,000 & Undirected \\
        \hline
        Bio-grid-fission-yeast \cite{nr} & 2,000 & 25,300 & Undirected \\
        \hline
    \end{tabular}
    \caption{Datasets used in the project}
    \label{tab:datasets}
\end{table}