\section*{2\quad  Datasets}
\addcontentsline{toc}{section}{2\quad Datasets}
The datasets that will be used in this project are nine and they are divided into three categories: Road Networks, Social Networks, and Biological Networks.
For each of these categories three datasets of different size were chosen: one small (\raisebox{-0.7ex}{\char`~}40k nodes), one medium (\raisebox{-0.7ex}{\char`~}100k nodes), and one large (\raisebox{-0.7ex}{\char`~}1M nodes).
The datasets, with their main properties (number of nodes, number of edges, and type of graph) are reported in the table \ref{tab:datasets}.
\begin{table}[h]
    \centering
    \begin{tabular}{|c|c|c|c|c|}
        \hline
        \textbf{\scalebox{0.87}{\large{Network}}} & \scalebox{0.87}{\large$\bm{|V|}$} & \scalebox{0.87}{\large$\bm{|E|}$} & \textbf{\scalebox{0.87}{\large{Type}}} \\
        \hline
        Pennsylvania \cite{nr} & 1.088.092 & 1.541.898 & Undirected \\
        \hline
        Padua (province) & 122.680 & 164.737 & Directed \\
        \hline
        Hong Kong (city) & 43.620 & 91.542 & Directed \\
        \hline
        Italian Covid-19 Retweet Network & 0 & 800.000 & Directed \\
        \hline
        Twitch & 168.114 & 6.797.557 & Undirected \\
        \hline
        GitHub & 37.700 & 289.003 & Undirected \\
        \hline
        Mus Musculus Protein Interactions & 0 & 800.000 & Undirected \\
        \hline
        Saccharomyces cerevisiae Protein Interactions & 0 & 100.000 & Undirected \\
        \hline
        Bio-grid-fission-yeast & 2000 & 25.300 & Undirected \\
        \hline
    \end{tabular}
    \caption{Datasets used in the project}
    \label{tab:datasets}
\end{table}