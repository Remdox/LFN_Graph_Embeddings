\section{Experiments}
The link prediction task requires a sample of edges from the graph (positive examples) and a sample of edges from the complement graph (negative example). Consequently, each dataset will be partitioned into training, validation, and test sets, in order to guarantee that each set contains an equal number of positive and negative examples.\\ \newline
Two metrics that are generally used for link prediction are the Area Under the Receiver Operating Characteristic curve (AUROC), and the Area Under the Precision-Recall curve (AUPR). AUROC is historically considered the primary performance metric used to evaluate the performances of link prediction methods \cite{Kalyani2025}. However, AUROC favors accurate classification of positive examples, at the cost of misclassifying the negative ones. In a scenario like the link prediction problem, which is inherently biased towards the negative class, this approach may not be optimal and can overestimate the performance of the methods. Consequently, the AUPR curve was also selected, as it can provide better evaluation of link prediction in the presence of class imbalance. \\ \newline
In regard to the implementation of the embedding techniques mentioned in Sec \ref{sec:methods}, the following implementations will be adopted:  \cite{Node2VecImpl} for Node2Vec, \cite{LINEImpl} for LINE, \cite{DVNEImpl} for DVNE, \cite{GraphSageImpl} for GraphSage. \\ \newline 
Finally, we are going to use Google Colab. The following are specifics of its basic runtime: Intel(R) Xeon(R) CPU 2-core @ 2.20GHz, 13 Gb, 110 Gb disk, Nvidia Tesla T4 (16 Gb VRAM). \\ \newline 